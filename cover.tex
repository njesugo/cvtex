\documentclass[11pt,a4paper]{letter}

\usepackage[french]{babel}
\usepackage[T1]{fontenc}
\usepackage[utf8]{inputenc}
\usepackage[sfdefault]{roboto} % Roboto en pdfLaTeX
\renewcommand{\familydefault}{\sfdefault}

\usepackage{geometry}
\usepackage{parskip}
\usepackage{microtype}
\usepackage[hidelinks]{hyperref}

\geometry{top=2.2cm,bottom=2.2cm,left=2.2cm,right=2.2cm}

\signature{Jesugo Aubin Nounagnon}

\begin{document}
\begin{letter}{\textbf{Sopra Steria}\\ Service Recrutement\\ Paris}
\date{\today}
\subject{Candidature -- Data Engineer}
\opening{Bonjour,}
\opening{Bonjour,}

Je vous écris pour candidater à un poste de \textbf{Data Engineer} chez \textbf{Sopra Steria}. Aujourd’hui, je cherche une équipe où je peux vraiment mettre les mains dans la data au quotidien : construire des pipelines solides, optimiser les performances, industrialiser proprement… et surtout livrer quelque chose d’utile, fiable, et compréhensible par les équipes métier.

Ces derniers mois, je travaille comme \textbf{Data Engineer chez Leroy Merlin / Groupe ADEO}. J’y ai développé une maîtrise avancée de \textbf{BigQuery} (optimisation des requêtes, structuration des tables, gestion des coûts) et j’ai aussi touché à l’\textbf{Infrastructure as Code} avec \textbf{Terraform}. Je suis à l’aise sur \textbf{GCP} et ses services (Compute, Storage, Networking, Dataflow, Composer/Airflow, Vertex AI). Ce que j’aime dans ce rôle, c’est le côté \emph{ingénierie} : partir d’un besoin, poser une architecture simple, automatiser, monitorer, et faire en sorte que ça tienne dans le temps. :contentReference[oaicite:0]{index=0}

Avant ça, j’étais \textbf{Data Steward (Data Gouvernance)} toujours chez ADEO, sur le périmètre \emph{Accounting to Cash}. J’y ai collaboré avec plus de 15 parties prenantes (Data Owners, Process Métier, Product Owners) et mis en place des contrôles de \textbf{data quality} (cohérence, doublons, valeurs manquantes/obsolètes) avec des tableaux de suivi et un reporting régulier. Résultat : j’ai pris un réflexe que je garde encore aujourd’hui : un pipeline, ce n’est pas juste ``ça tourne'', c’est ``on peut faire confiance aux données''. Et sur des projets data à forte exposition, ça change tout. :contentReference[oaicite:1]{index=1}

Côté dev, je suis à l’aise avec \textbf{Python}, \textbf{SQL} et \textbf{Bash}, et je garde une culture \emph{delivery} : Agile, esprit d’équipe, proactivité, et communication claire (même quand c’est technique). J’ai envie de rejoindre Sopra Steria parce que c’est exactement le genre d’environnement où je peux monter en puissance : diversité des clients, sujets ambitieux, et exigences réelles sur la qualité, la sécurité et l’industrialisation.

Si mon profil vous parle, je serais content d’échanger avec vous. Je peux vous donner des exemples concrets de pipelines/optimisations réalisés, et expliquer comment j’aborde la fiabilité, les coûts et la mise en production.
\closing{Bien cordialement,}
\end{letter}
\end{document}
